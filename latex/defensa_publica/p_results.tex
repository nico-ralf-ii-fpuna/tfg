\section{Pruebas y resultados}



\subsection{Conjuntos de datos de prueba}

\begin{frame}
    \frametitle{Conjuntos de datos utilizados}

    \begin{itemize}
        \item
        Conjuntos utilizados:

        \begin{itemize}
            \item
            CSIC 2010\footnote{http://www.isi.csic.es/dataset/}

            \item
            CSIC TORPEDA 2012\footnote{http://www.tic.itefi.csic.es/torpeda}
        \end{itemize}

        \item
        Peticiones HTTP simuladas a una aplicación de comercio electrónico

        \item
        Distintos tipos de ataques

        \begin{itemize}
            \item
            Inyección SQL, \textit{buffer overflow}, \textit{cross-site scripting} (XSS),
            entre otros
        \end{itemize}

        \item
        Datos utilizados:

        \begin{itemize}
            \item
            18 grupos de peticiones según método HTTP y URL

            \item
            \num{40130} peticiones normales y \num{42444} anomalías
        \end{itemize}
    \end{itemize}
\end{frame}



\subsection{Análisis de la eficacia de detección}

\begin{frame}
    \frametitle{Pruebas de eficacia de detección}

    \begin{center}
        \includegraphics[width=\textwidth, trim=1.1cm 1.7cm 1.6cm 1cm]{images/diagram-score-explanation.pdf}
    \end{center}

    $$
    \text{TPR} = \frac{\text{TP}}{\text{P}}
    \quad , \ \quad
    \text{FPR} = \frac{\text{FP}}{\text{N}}
    \quad , \ \quad
    \text{F}_{1}\textit{-score} = \frac{2 \text{TP}}{2 \text{TP} + \text{FP} + \text{FN}}
    $$

    \begin{itemize}
        \item
        \small $\text{F}_{1}$-\textit{score} es más robusto frente a datos no balanceados
    \end{itemize}
\end{frame}

\begin{frame}
    \frametitle{Pruebas de eficacia de detección}

    \begin{itemize}
        \item
        Mejoras obtenidas por el análisis de valores de parámetros

        \begin{itemize}
            \item
            Promedio de los 18 grupos

            \item
            3 iteraciones con distintos subconjuntos de entrenamiento

            \item
            Entrenamiento:

            \begin{itemize}
                \item
                \num{1500} peticiones normales en cada grupo

                \item
                $\approx 75\%$ de los datos normales
            \end{itemize}

            \item
            Detección:

            \begin{itemize}
                \item
                $\approx$ \num{2000} peticiones normales en cada grupo

                \item
                $\approx$ \num{1300} peticiones anómalas en cada grupo
            \end{itemize}
        \end{itemize}
    \end{itemize}

    \begin{center}
        \small
        \begin{tabular}{|l|c|c|c|}
            \hline
                                       & TPR             & FPR             & F$_{1}$-\textit{score} \\ \specialrule{1.5pt}{0}{0}
            Solo petición completa     & $0.77 \pm 0.28$ & $0.11 \pm 0.22$ & $0.79 \pm 0.23$        \\ \hline
            Con análisis de parámetros & $0.93 \pm 0.11$ & $0.03 \pm 0.03$ & $0.95 \pm 0.08$        \\ \hline
        \end{tabular}
    \end{center}
\end{frame}



\subsection{Análisis del tiempo de respuesta de las aplicaciones}

\begin{frame}
    \frametitle{Pruebas de tiempo de respuesta}

    \begin{itemize}
        \item
        Medición del impacto de OCS-WAF sobre las aplicaciones protegidas

        \begin{itemize}
            \item
            Promedio de los 18 grupos

            \item
            100 peticiones de cada grupo
        \end{itemize}
    \end{itemize}

    \begin{center}
        \small
        \begin{tabular}{|l|r|}
            \hline
                                        & \multicolumn{1}{c|}{Tiempo de respuesta promedio} \\
                                        & \multicolumn{1}{c|}{(en milisegundos)}            \\
            \specialrule{1.5pt}{0}{0}
            Directo a la aplicación web & \num{4.0} $\pm$ \num{0.6}                         \\ \hline
            A través de OCS-WAF         & \num{8.7} $\pm$ \num{1.3}                         \\ \hline
        \end{tabular}
    \end{center}
\end{frame}




\subsection{Compración con otros trabajos}

\begin{frame}
    \frametitle{Trabajos relacionados de otros autores}

    \begin{itemize}
        \item
        Utilización del mismo conjunto de datos CSIC 2010
    \end{itemize}

    \begin{center}
        \small
        \begin{tabular}{|l|l|c|c|c|}
            \hline
            \multicolumn{1}{|c|}{Propuesta} & \multicolumn{1}{c|}{Observaciones} & TPR & FPR & F$_{1}$-\textit{score} \\
            \specialrule{1.5pt}{0}{0}
            OCS-WAF
            & \footnotesize One-Class SVM
            & \num{0.93}
            & \num{0.03}
            & \num{0.95} \\ \hline
            ModSecurity%
                \footnote{https://www.modsecurity.org/}
                \footnote{Giménez (2015) HTTP-WS-AD: Detector de
                    anomalías orientada a aplicaciones web y web services.}
            & \footnotesize Detección por firmas
            & \alert{\num{0.56}}
            & \textcolor[RGB]{10,128,0}{\num{0.00}}
            & \alert{\num{0.71}} \\ \hline
            HTTP-WS-AD%
                \footnote{Giménez (2015) HTTP-WS-AD: Detector de
                    anomalías orientada a aplicaciones web y web services.}
            & \footnotesize Métodos estadísticos
            & \textcolor[RGB]{10,128,0}{\num{0.99}}
            & \textcolor[RGB]{10,128,0}{\num{0.02}}
            & \textcolor[RGB]{10,128,0}{\num{0.99}} \\ \hline
            Torrano-Giménez%
                \footnote{Torrano-Giménez (2015) Study of stochastic and
                    machine learning techniques for anomaly-based detection.}
            & \footnotesize Árboles de decisión
            & \textcolor[RGB]{10,128,0}{\num{0.95}}
            & \alert{\num{0.05}}
            & -          \\ \hline
            OC-WAD%
                \footnote{Parhizkar y Abadi (2015) OC-WAD: A one-class
                    classifier ensemble approach for anomaly detection.}
            & \footnotesize One-Class SVM
            & \textcolor[RGB]{10,128,0}{\num{0.96}}
            & \num{0.03}
            & -          \\ \hline
        \end{tabular}
    \end{center}
\end{frame}
