\section{Conclusiones}



\subsection{Resumen de la investigación}

\begin{frame}
    \frametitle{OCS-WAF}

    \begin{itemize}
        \item
        Implementación de un WAF para protección de múltiples aplicaciones
        web

        \item
        Detección de anomalías en mensajes HTTP mediante clasificadores
        One-Class SVM
    \end{itemize}

    \begin{center}
        \includegraphics[width=\textwidth]{images/waf-diagram-overview.png}
    \end{center}
\end{frame}

\begin{frame}
    \frametitle{Aporte de nuestra investigación}

    \begin{itemize}
        \item
        Combinación de procesos de extracción de \textit{features} para
        peticiones HTTP con clasificadores One-Class SVM

        \begin{itemize}[<.->]
            \item
            Análisis de valores de parámetros

            \item
            Descripciones más precisas de los valores normales
        \end{itemize}

        \item<2->
        Código fuente de OCS-WAF

        \begin{itemize}[<.->]
            \item
            Incentivar la reproducción de nuestros resultados

            \item
            Facilitar la extensión de nuestro trabajo
        \end{itemize}
    \end{itemize}
\end{frame}



\subsection{Logro de los objetivos}

\begin{frame}
    \frametitle{Objetivos específicos}

    \begin{enumerate}[<+(1)->]
        \item
        Diseñar procesos de extracción de características (\textit{features})
        específicamente para mensajes HTTP, basado en aportes de otros
        investigadores de la literatura.

        \begin{itemize}[<.->]
            \item
            Diseño de nuevos procesos de extracción de \textit{features}
            para mensajes HTTP
        \end{itemize}

        \item
        Implementar un WAF basado en anomalías, utilizando los procesos de
        extracción de \textit{features} diseñados junto con clasificadores
        One-Class SVM.

        \begin{itemize}[<.->]
            \item
            Implementación de OCS-WAF
        \end{itemize}

        \savemynewenumi
    \end{enumerate}
\end{frame}

\begin{frame}
    \frametitle{Objetivos específicos}

    \begin{enumerate}[<+->]
        \contmynewenumi

        \item
        Evaluar la eficacia del WAF implementado en cuanto a la detección
        de mensajes HTTP anómalos.

        \begin{itemize}[<.->]
            \item
            Pruebas de eficacia de detección con un
            TPR de \num{0.93},
            FPR de \num{0.03} y
            F$_{1}$-\textit{score} de \num{0.95}
        \end{itemize}

        \item
        Analizar la viabilidad de utilizar el WAF implementado para
        detección de ataques en tiempo real.

        \begin{itemize}[<.->]
            \item
            Realización de pruebas de impacto de OCS-WAF sobre el tiempo
            de respuesta de las aplicaciones protegidas
        \end{itemize}
    \end{enumerate}
\end{frame}

\begin{frame}
    \frametitle{Objetivo general}

    \begin{itemize}[<+(1)->]
        \item
        Detectar mensajes HTTP anómalos entre las aplicaciones web y
        sus usuarios con el fin de mitigar los riesgos de ataques contra
        dichas aplicaciones, utilizando un WAF basado en clasificadores
        One-Class SVM.

        \begin{itemize}[<.->]
            \item
            Detección de mensajes HTTP anómalos con OCS-WAF
        \end{itemize}
    \end{itemize}
\end{frame}



\subsection{Trabajos futuros}

\begin{frame}
    \frametitle{Ideas para trabajos futuros}

    \begin{itemize}[<+(1)->]
        \item
        Realizar pruebas con otros conjuntos de datos.

        \item
        Explorar otras características de los mensajes HTTP.

        \item
        Explorar otras herramientas (en vez del One-Class SVM) para la
        detección.

        \item
        Extender para incluir cuerpos de otros formatos, por ejemplo
        datos binario, JSON, XML, entre otros.

        \item
        Extender para incluir mensajes HTTP/2.

        \item
        Explorar la posibilidad de paralelizar el proceso de entrenamiento
        en OCS-WAF.
    \end{itemize}
\end{frame}



\subsection{Publicaciones}

\begin{frame}
    \frametitle{Publicación de nuestro trabajo}

    \begin{itemize}
        \footnotesize

        \item
        \textbf{Título:}
        Anomaly-based Web Application Firewall using HTTP-specific features
        and One-Class SVM
    \end{itemize}

    \begin{block}{\small WRSeg 2017}
        \begin{columns}
            \column{0.6\textwidth}
            \begin{itemize}
                \footnotesize

                \item
                Workshop Regional de Segurança da Informação e
                Sistemas Computacionais

                \item
                Santa María, Brasil

                \item
                25 de Setiembre 2017
            \end{itemize}

            \column{0.4\textwidth}
            \begin{center}
                \includegraphics[width=\textwidth]{images/logo-errc2017.png}
            \end{center}
        \end{columns}
    \end{block}

    \begin{block}{\small Revista REABTIC}
        \begin{columns}
            \column{0.6\textwidth}
            \begin{itemize}
                \footnotesize

                \item
                Revista Eletrônica Argentina-Brasil de Tecnologias da Informação
                e da Comunicação

                \item
                Enviado y en revisión
            \end{itemize}

            \column{0.4\textwidth}
            \begin{center}
                \includegraphics[width=\textwidth]{images/logo-reabtic.png}
            \end{center}
        \end{columns}
    \end{block}
\end{frame}
