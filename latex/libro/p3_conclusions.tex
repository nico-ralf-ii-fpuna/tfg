\renewcommand{\newCommandChapterTitle}{Conclusiones}
\chapter{\newCommandChapterTitle}
\markright{\hfill \thechapter. \newCommandChapterTitle}
\label{chap:p3_conclusions}


En este capítulo presentamos las conclusiones de nuestro trabajo.
Primeramente damos un resumen general del contenido que fue presentado
a lo largo de este trabajo, incluyendo el análisis de los resultados
que obtuvimos en las pruebas realizadas con el detector \gls{acr3:name}.
Después mostramos de qué forma hemos logrado alcanzar los objetivos que
nos propusimos para esta investigación. Finalmente, concluimos este
trabajo describiendo varios aspectos de nuestra implementación en los
cuales pueden realizarse mejoras; estos aspectos podrían dar lugar a
trabajos futuros.


\section{Resumen de la investigación}

En esta sección presentamos los aspectos más importantes que fueron
descritos en cada uno de los capítulos anteriores, buscando brindar de
esta forma un resumen de las ideas principales de este trabajo.

En el \autoref{chap:p3_introduction} introducimos el tema de este trabajo.
Describimos el área de estudio que rodea nuestra investigación, que abarca
el área de \gls{acr3:ids} con detección de anomalías, características
de los mensajes \gls{acr3:http} y también la herramienta \gls{acr3:ocsvm}
del área de \gls{acr3:ml}. Luego mostramos la problemática que nos llevó
a realizar este trabajo, argumentando que vulnerabilidades presentes en
muchas aplicaciones web presentan un gran riesgo y se necesita mecanismos
para protegerlos. Al final de ese capítulo presentamos los objetivos que
nos hemos propuesto para este trabajo de investigación; estos objetivos
se centran en la detección de mensajes \gls{acr3:http} anómalos mediante
un \gls{acr3:waf} basado en detección de anomalías, utilizando clasificadores
\gls{acr3:ocsvm}, con el fin de mitigar los riesgos existentes de ataques
contra aplicaciones web a través de la Internet.

En el \autoref{chap:p3_concepts_features} presentamos nuestros procesos
de extracción de \textit{features}. Estos procesos se basan de gran manera
en los aportes de los trabajos de Kruegel y Vigna, especialmente en la
manera como aprovechan las características de mensajes \gls{acr3:http}
para proponer sus modelos de anomalías. Luego explicamos los procesos que
diseñamos, incluyendo el análisis de valores de parámetros que realizamos
y cada uno de los \textit{features} que extraemos de esos valores para
representar las características de los mensajes como vectores numéricos.
Estos vectores pueden tener longitudes distintas según el grupo de método
\gls{acr3:http} y \gls{acr3:url} al que pertenecen, ya que esto depende
de la cantidad de parámetros de las peticiones.
Las características analizadas de las peticiones son la distribución de
caracteres, la entropía y la cantidad de caracteres. Se complementa esta
extracción de \textit{features} con un proceso de escalamiento para reducir
el impacto de los distintos rangos de números que pueden tener los diferentes
\textit{features}.

En el \autoref{chap:p3_concepts_ocsvm} describimos en detalle el
clasificador \gls{acr3:ocsvm}, que constituye la parte central de nuestro
\gls{acr3:waf}. Una de las ventajas de utilizar este clasificador es que
no necesitamos peticiones anómalas para el entrenamiento, reduciendo así
el impacto que tiene la aparición de nuevos tipos de ataques; solamente
se vuelve a realizar el entrenamiento cuando haya cambios en las aplicaciones
protegidas.
Mostramos que se utiliza vectores de \textit{features} que representan
a las peticiones de entrenamiento para trazar un hiperplano que separa
este grupo de entrenamiento del origen. Posteriormente, en la fase de
detección, se determina a qué lado del hiperplano pertenecen los vectores
de las nuevas peticiones, para así determinar si corresponden a peticiones
normales o anómalas.
La selección de valores de los parámetros \gls{sim3:nui} y \gls{sim3:gammai}
del clasificador se realizó de forma experimental para los conjuntos de
datos utilizados y creemos que esto no variará mucho para conjuntos de
datos diferentes.

El \autoref{chap:p3_new_waf} detalla la implementación y funcionamiento
de \gls{acr3:name} que fue construido en el marco de este trabajo.
Describimos los detalles de la implementación, incluyendo los pasos
intermedios de cada una de las dos fases, que son las fases de entrenamiento
y detección. Se explicaron las particularidades de nuestra implementación,
detallando también los componentes externos y librerías que utilizamos.
El código fuente de la implementación está disponible en un repositorio
público bajo la dirección \TheRepoUrl.

El \autoref{chap:p3_results} presenta las pruebas realizadas con nuestra
implementación y expone los resultados obtenidos en cada una de ellas.
Utilizamos los conjuntos de datos \gls{acr3:csic} 2010 \citep{csic2010dataset}
y \gls{acr3:csic} TORPEDA 2012 \citep{torpeda2012dataset}, que contienen
peticiones \gls{acr3:http} hechas a una aplicación de comercio electrónico.
Las pruebas experimentales analizaron tres características de \gls{acr3:name}:
la eficacia de detección, el impacto en el tiempo de respuesta de las
aplicaciones protegidas y el tiempo de entrenamiento.

Analizando la primera de estas característica, la eficacia de detección
de \gls{acr3:name}, logramos obtener un \gls{acr3:tpr} de \num{0.93},
un \gls{acr3:fpr} de \num{0.03} y un F$_{1}$-\textit{score} de \num{0.95}.
Estos son resultados promedios para los 18 grupos de peticiones que
utilizamos de los conjuntos de datos.
Empleamos el escalamiento de \textit{features} e incluimos el análisis
de valores de parámetros en las peticiones, ya que utilizando estas dos
estrategias obtuvimos los mejores resultados. Además, realizamos el
entrenamiento con 1500 peticiones en cada grupo (cerca del 75\% de las
muestras normales), atendiendo a que no haya anomalías en los datos de
entrenamiento.
Notamos que el \gls{acr3:tpr} promedio alcanzado de \num{0.93} es algo
bajo; inspeccionando los grupos individuales descubrimos que la mayoría
tiene un \gls{acr3:tpr} mayor o igual a \num{0.99}, pero cuatro grupos
alcanzan solamente un \gls{acr3:tpr} alrededor de \num{0.72}. Estos cuatro
grupos justamente son los que tienen la mayor cantidad de parámetros (y
por lo tanto la mayor cantidad de \textit{features}), y esto podría indicar
que nuestro \gls{acr3:waf} tiene dificultades con peticiones que tienen
muchos parámetros.

En la segunda característica analizada mediante las pruebas, tratamos
de cuantificar el impacto que nuestro \gls{acr3:waf} podría tener sobre
el tiempo de respuesta de las aplicaciones web que protege. A pesar de
que nuestra implementación puede ser optimizada, logramos que el retraso
adicional en el tiempo de repuesta sea despreciable comparado con la
latencia promedio del tráfico de red en Internet, por lo que el trabajo
de detección de nuestro \gls{acr3:waf} no afectaría de forma notable
el tiempo de respuesta de las aplicaciones protegidas.

En la tercera característica analizada mediante las pruebas, que trata
del tiempo de entrenamiento de nuestro \gls{acr3:waf}, verificamos que
esta duración depende de la cantidad de peticiones utilizadas para dicha
fase de entrenamiento. Los números indican que se puede entrenar nuestra
implementación con \num{100000} peticiones en alrededor de 12 minutos.
Esta rapidez le provee mucha flexibilidad a los administradores del
\gls{acr3:waf} para poder planificar los momentos de entrenamiento.
Recordamos que el entrenamiento solamente es necesario después de realizar
cambios en las aplicaciones protegidas, y la eficacia de detección del
sistema no debería verse afectada por la aparición de nuevos ataques.

Al final del \autoref{chap:p3_results} realizamos también una comparación
de \gls{acr3:name} con propuestas de otros trabajos del área. Para poder
comparar los resultados, nos enfocamos en aquellos que utilizan el mismo
conjunto de datos que nosotros empleamos para las pruebas. Analizamos
propuestas que utilizan distintas herramientas, como por ejemplo árboles
de decisión, modelos estadísticos y también el mismo clasificador
\gls{acr3:ocsvm}.


\section{Alcance de los objetivos}

En esta parte volvemos a presentar los objetivos que nos habíamos propuesto
en el \autoref{chap:p3_introduction} y mostramos de qué manera logramos
alcanzar los mismos a través del trabajo realizado a lo largo de esta
investigación. Empezamos describiendo nuestros logros para los objetivos
específicos, para después cerrar con el objetivo general.


\subsection{Objetivos específicos}

\begin{enumerate}
    \item
    Diseñar procesos de extracción de características (\textit{features})
    específicamente para mensajes \gls{acr3:http}, basado en aportes de
    otros investigadores de la literatura.

    \begin{itemize}
        \item
        Partiendo de los trabajos de los autores Kruegel y Vigna, y
        combinando sus aportes con trabajos de la literatura especializada,
        se diseñó procesos de extracción de \textit{features} que pueden
        representar características de los mensajes \gls{acr3:http}
        mediante vectores de \textit{features} numéricos.
        Estos procesos de extracción fueron descritos en el
        \autoref{chap:p3_concepts_features}, mientras que en el
        \autoref{chap:p3_new_waf} se mostraron los detalles de la
        implementación de los mismos.
        Las pruebas experimentales presentadas en el \autoref{chap:p3_results}
        demuestran la utilidad de estos procesos diseñados para la tarea
        de detección de anomalías en mensajes \gls{acr3:http}.
    \end{itemize}

    \item
    Implementar un \gls{acr3:waf} basado en anomalías, utilizando los
    procesos de extracción de \textit{features} diseñados junto con
    clasificadores \gls{acr3:ocsvm}.

    \begin{itemize}
        \item
        Se implementó un \gls{acr3:waf} sencillo, utilizando los procesos
        de extracción de \textit{features} que fueron descritos en el
        \autoref{chap:p3_concepts_features}, junto con los clasificadores
        \gls{acr3:ocsvm} que fueron presentados en el
        \autoref{chap:p3_concepts_ocsvm}.
        Los detalles de implementación del \gls{acr3:name} fueron explicados
        en el \autoref{chap:p3_new_waf}. Se incluyó la referencia al
        repositorio público que contiene el código fuente del \gls{acr3:name}.
    \end{itemize}

    \item
    Evaluar la eficacia del \gls{acr3:waf} implementado en cuanto a la
    detección de mensajes \gls{acr3:http} anómalos.

    \begin{itemize}
        \item
        Mediante las pruebas del \autoref{chap:p3_results} se mostró
        que el \gls{acr3:name} es eficaz en la detección de mensajes
        \gls{acr3:http} anómalos, considerando los conjuntos de datos
        de prueba utilizados en este trabajo.
    \end{itemize}

    \item
    Analizar la viabilidad de utilizar el \gls{acr3:waf} implementado
    para detección de ataques en tiempo real.

    \begin{itemize}
        \item
        Utilizando la implementación sencilla, con las pruebas del
        \autoref{chap:p3_results} se mostró que el \gls{acr3:name} no
        afectaría de forma notable el tiempo de respuesta de las
        aplicaciones web que están siendo protegidas por el mismo,
        posibilitando que pueda ser utilizado para detección de
        ataques en tiempo real.
    \end{itemize}
\end{enumerate}


\subsection{Objetivo general}

\begin{itemize}
    \item
    Detectar mensajes \gls{acr3:http} anómalos entre aplicaciones web
    y sus usuarios con el fin de mitigar los riesgos de ataques contra
    dichas aplicaciones, utilizando un \gls{acr3:waf} basado en
    clasificadores \gls{acr3:ocsvm}.

    \begin{itemize}
        \item
        Se construyó un \gls{acr3:waf} que usa características
        específicas de mensajes \gls{acr3:http} y clasificadores
        \gls{acr3:ocsvm} para detectar mensajes anómalos.
        \gls{acr3:name} puede ser colocado frente a múltiples
        aplicaciones web con el fin de protegerlas contra posibles
        ataques.
        De esta forma, en nuestra opinión, se ha logrado cumplir
        satisfactoriamente con todos los objetivos propuestos para
        este trabajo.
    \end{itemize}
\end{itemize}


\section{Ideas para trabajos futuros}

A pesar de concluir exitosamente nuestra investigación, reconocemos que
aún hay varios aspectos que pueden ser mejorados o extendidos. A continuación
presentamos algunas de estas ideas que podrían dar lugar a trabajos futuros.

\begin{itemize}
    \item
    Se podría realizar pruebas con \gls{acr3:name} utilizando otros
    conjuntos de datos. Esto serviría para verificar y validar que nuestra
    implementación no está atada solamente a los datos que utilizamos en
    este trabajo.

    \item
    Nuestros resultados aún dejan lugar para mejoras, especialmente el
    \gls{acr3:tpr} es un poco bajo para algunos casos. Se podría explorar
    otras características de los mensajes \gls{acr3:http} para extender y
    mejorar nuestros procesos de extracción de \textit{features}, analizando
    el impacto que tiene cada característica en la detección.

    \item
    Actualmente nuestra implementación solamente considera los cuerpos
    de peticiones que constan de pares de parámetros y valores estructurados
    de la misma forma que el \textit{query string}. Se puede extender el
    \gls{acr3:name} para incluir cuerpos de otros formatos, por ejemplo
    datos binario, JSON, XML, entre otros.
    Adicionalmente, se necesita encontrar o construir un conjunto de datos
    de prueba que contenga ese tipo de peticiones con datos suficientemente
    reales.

    \item
    \gls{acr3:name} solamente considera mensajes \gls{acr3:http} en la
    versión 1.1 del protocolo. En la versión 2 de este protocolo, dichos
    mensajes son enviados en formato binario \citep{belshe2015http2} % from section 1
    y se podría incluir una extensión a nuestra implementación para poder
    analizar también los mensajes \gls{acr3:http}/2.

    \item
    La selección de los valores para los parámetros \gls{sim3:nui} y
    \gls{sim3:gammai} presenta un gran desafío y \gls{acr3:name} no
    cuenta todavía con un método automático para esta búsqueda.
    Se podría explorar métodos para realizar la selección óptima de
    estos parámetros, para posteriormente incorporar también ese
    mecanismo descubierto dentro de nuestra implementación.
\end{itemize}
