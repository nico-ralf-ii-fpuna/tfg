\begin{abstracts}
    Vulnerabilities in web applications pose a great risk because they
    can be exploited by malicious attackers through the Internet. Web
    Application Firewalls (\gls{acr3:waf}) can be placed in front of
    these applications to detect possible attacks, thus reducing the
    impact of these risks.
    In this work, we present \gls{acr3:name}, a \gls{acr3:waf} that can
    be placed in front of web applications to analyze the incoming
    \gls{acr3:http} messages, in order to detect anomalous messages that
    could contain attacks.
    Our implementation, made with the \textit{Python} programming language,
    uses \gls{acr3:ocsvm} classifier for the detection, coupled with custom
    \gls{acr3:http}-specific feature extraction processes.
    \gls{acr3:name} is trained with messages that represent the normal
    behavior of the protected applications, and later, during the detection
    phase, it can detect anomalous messages or attacks.
    With anomaly detection strategy, \gls{acr3:name} only needs to be
    retrained when there are changes in the protected applications, but
    the discovery of new attacks does not require this retraining.
    Our detection efficacy tests show that \gls{acr3:name} reaches an
    average \gls{acr3:tpr} of \num{0.93}, an average \gls{acr3:fpr} of
    \num{0.03} and an average F$_{1}$-\textit{score} of \num{0.95} for
    the public data sets that we used.
    The tests that we applied also show that the detection process of
    \gls{acr3:name} should not have a noticeable effect on the response
    time of the protected applications. Besides, it can be trained with
    \num{100000} normal messages in only a few minutes.
    Finally, the source code of \gls{acr3:name} is available in our
    public repository under \TheRepoUrl, so that others may reproduce
    our results and extend our work with further research.

    \keywordsENG{
        Intrusion Detection System (\gls{acr3:ids}),
        Web Application Firewall (\gls{acr3:waf}),
        web attacks,
        Anomaly Detection,
        \gls{acr3:ocsvm}}
\end{abstracts}
