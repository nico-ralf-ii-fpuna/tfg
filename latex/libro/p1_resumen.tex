\begin{resumen}
    Las vulnerabilidades en aplicaciones web presentan un gran riesgo, ya
    que estas pueden ser explotadas por atacantes maliciosos a través
    de Internet. Los \textit{Web Application Firewalls} (\gls{acr3:waf})
    pueden ser colocados frente a estas aplicaciones para detectar
    posibles ataques y de esta forma reducir estos riesgos.
    En este trabajo presentamos \gls{acr3:name}, un \gls{acr3:waf} que
    puede ser colocado frente a aplicaciones web para analizar los mensajes
    \gls{acr3:http} entrantes, con el fin de detectar mensajes anómalos
    que podrían contener ataques.
    Nuestra implementación, hecha en el lenguage de programación
    \textit{Python}, utiliza clasificadores \gls{acr3:ocsvm} para la
    detección, junto con procesos de extracción de características
    diseñados específicamente para mensajes \gls{acr3:http}.
    \gls{acr3:name} es entrenado con mensajes que representan el uso
    normal de las aplicaciones protegidas, y posteriormente, en la fase
    de detección, puede detectar mensajes anómalos o ataques.
    Usando esta estrategia de detección de anomalías, \gls{acr3:name}
    solamente necesita ser entrenado cuando haya cambios en las
    aplicaciones protegidas, por lo que la aparición de nuevos tipos de
    ataques no requiere volver a entrenarlo.
    Las pruebas realizadas para medir la eficacia de detección muestran
    que \gls{acr3:name} alcanza un \gls{acr3:tpr} promedio de \num{0.93},
    un \gls{acr3:fpr} promedio de \num{0.03} y un F$_{1}$-\textit{score}
    promedio de \num{0.95} para los conjuntos de datos públicos que
    utilizamos.
    Las pruebas también evidencian que las tareas de detección de \gls{acr3:name}
    no afectarían de forma notable el tiempo de respuesta de las aplicaciones
    protegidas. Además, puede ser entrenado con \num{100000} mensajes
    normales en unos pocos minutos.
    Finalmente, el código fuente de \gls{acr3:name} está disponible en
    un repositorio público bajo la dirección \TheRepoUrl, con la finalidad
    de que otros puedan reproducir nuestros resultados y extender este
    trabajo en futuras investigaciones.

    \keywordsESP{
        Sistemas de Detección de Intrusión (\gls{acr3:ids}),
        Web Application Firewall (\gls{acr3:waf}),
        ataques web,
        detección de anomalías,
        \gls{acr3:ocsvm}}
\end{resumen}
